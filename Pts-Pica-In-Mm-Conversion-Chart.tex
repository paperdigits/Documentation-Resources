\documentclass[12pt, letterpaper, oneside]{article}

\usepackage{longtable}
\usepackage{colortbl}
%------------Page Demnsions------------%
\addtolength{\oddsidemargin}{-0.5in} %Sets the odd numbered page margins to 1 inch.
\addtolength{\evensidemargin}{-0.5in} %Sets the even numbered page margins to 1 inch.
\addtolength{\textwidth}{1in} %Widens the text so that flows from margin to margin. This number is double the values set above if odd & even side margins are the same.
\setlength\headheight{.25in} %Sets the height of the header to 1/4 of an inch.
\addtolength{\topmargin}{-.875in} %Sets the top margin to 3/4 of an inch.
\addtolength{\textheight}{1in} %Sets the length from the top margin for the text to start.
\addtolength{\headheight}{.25in}
%------------Page Demnsions------------%
%------------Headers/Footers------------%
\usepackage{fancyhdr} %Package to allow fine grain control of the headers and footers of documents.
\pagestyle{fancy} %Sets the page styles to fancy.

\fancyhead{} %Clears LaTeX's default header style. Do not add any text here.
\fancyfoot{} %Clears LaTeX's default footer style. Do not add any text here.
			
\lhead{} %The text in the left hand part of header.
\chead{\Large \sffamily Conversion Chart -- Picas, Points, Inches, \& Millimeters} %The text in the center of the header.
\rhead{} %The text in the right hand part of the header.
\lfoot{\small \sffamily \today} %The text in the left hand part of the footer.
\cfoot{} %The text in the center part of the footer.
\rfoot{} %The text in the right hand part of the footer.
%------------Headers/Footers------------%

%------------Fonts------------%
%\usepackage{palatino}
\usepackage{helvet}
\usepackage{sectsty}
\allsectionsfont{\normalfont \sffamily}
%------------Fonts------------%
%------------Multicolumn TOC------------%
\RequirePackage{multicol}
%\RequirePackage{ifthen}
%\usepackage[toc]{multitoc}
%------------Multicolumn TOC------------%

\begin{document}
\sffamily

\begin{center}
\begin{tabular}{|l|l|l|l|l|}
\hline
\textbf{Inches} & \textbf{Pica} & \textbf{Pica (common)} & \textbf{Points} & \textbf{Points (common)} \tabularnewline \hline
1 & 6.022 & 6 & 72.270 & 72 \tabularnewline \hline
\rowcolor[rgb]{0.85,0.85,0.85} 1.25 & 7.528 & 7.5 & 90.337 & 90 \tabularnewline \hline
1.50 & 9.033 & 9 & 108.405 & 108 \tabularnewline \hline
\rowcolor[rgb]{0.85,0.85,0.85} 1.75 & 10.539 & 10.5 & 126.472 & 126 \tabularnewline \hline
2.00 & 12.044 & 12 & 144.540 & 144 \tabularnewline \hline
\end{tabular}
\end{center}

\noindent 1 ft = 12 inches = 72 picas\\
1 in = 6 picas = 72 points\\
1 pica = 12 points

\begin{center}
\begin{tabular}{|l|l|l|}
\hline
\textbf{Inches} & \textbf{Points} & \textbf{Picas,Points} \tabularnewline \hline 
.25 & 18 & 1p6 \tabularnewline \hline  
\rowcolor[rgb]{0.85,0.85,0.85} .50 & 36 & 3p \tabularnewline \hline
.75 & 54 & 4p6 \tabularnewline \hline
\end{tabular}
\end{center}

\noindent 1.25 inches = 7p6 (7 picas and 6 points)\\
1.50 inches = 9p\\
1.75 inches = 10p6


\end{document}