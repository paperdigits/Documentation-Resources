\documentclass[12pt, letterpaper, oneside]{article}

\usepackage{amsfonts}

%------------Page Demnsions------------%
\addtolength{\oddsidemargin}{-0.5in} %Sets the odd numbered page margins to 1 inch.
\addtolength{\evensidemargin}{-0.5in} %Sets the even numbered page margins to 1 inch.
\addtolength{\textwidth}{1in} %Widens the text so that flows from margin to margin. This number is double the values set above if odd & even side margins are the same.
\setlength\headheight{.25in} %Sets the height of the header to 1/4 of an inch.
\addtolength{\topmargin}{-.875in} %Sets the top margin to 3/4 of an inch.
\addtolength{\textheight}{1in} %Sets the length from the top margin for the text to start.
\addtolength{\headheight}{.25in}
%------------Page Demnsions------------%
%------------Fonts------------%
%\usepackage{palatino}
\usepackage{helvet}
\usepackage{sectsty}
\allsectionsfont{\normalfont \sffamily}
%------------Fonts------------%
%------------Multicolumn TOC------------%
\RequirePackage{multicol}
%\RequirePackage{ifthen}
%\usepackage[toc]{multitoc}
%------------Multicolumn TOC------------%
%------------Headers/Footers------------%
\usepackage{fancyhdr} %Package to allow fine grain control of the headers and footers of documents.
\pagestyle{fancy} %Sets the page styles to fancy.

\fancyhead{} %Clears LaTeX's default header style. Do not add any text here.
\fancyfoot{} %Clears LaTeX's default footer style. Do not add any text here.
			
\lhead{} %The text in the left hand part of header.
\chead{\Large \sffamily Checklist Summary} %The text in the center of the header.
\rhead{} %The text in the right hand part of the header.
\lfoot{\small \sffamily \today} %The text in the left hand part of the footer.
\cfoot{} %The text in the center part of the footer.
\rfoot{} %The text in the right hand part of the footer.
%------------Headers/Footers------------%

\begin{document}
\begin{multicols}{2}
\sffamily

\paragraph{Our manual:}

\begin{itemize} \itemsep -2pt
	\item[$\Box$] complies with all labeling regulations
	\item[$\Box$]  is written for the type of people who use our device
	\item[$\Box$]  tells the user how to get help from us
	\item[$\Box$]  includes a Table of Contents
	\item[$\Box$]  has general warnings and precautions at he beginning
	\item[$\Box$]  briefly describes the purpose of the device
	\item[$\Box$]  give a physical description of the device with a graphic
	\item[$\Box$]  explains conditions under which the device should and should not be used
	\item[$\Box$]  gives clear setup instructions
	\item[$\Box$]  gives clear check-out procedures
	\item[$\Box$]  gives clear and easy to follow operating instructions
	\item[$\Box$]  provides cleaning instructions
	\item[$\Box$]  describes maintenance that the lay user should do
	\item[$\Box$]  explains storage
	\item[$\Box$]  has a clear, easy-to-use \& find troubleshooting section
	\item[$\Box$]  has a summary page with all the critical information on it
	\item[$\Box$]  has an alphabetized index
	\item[$\Box$]  has an easy to find date of printing
	\item[$\Box$]  includes instructions on any accessories
\end{itemize}

\paragraph{We have:}

\begin{itemize} \itemsep -2pt
	\item[$\Box$]  done a task analysis for the procedures in our manual
	\item[$\Box$]  selected a suitable format (text, flowchart, list)
	\item[$\Box$]  written and formatted procedures correctly
	\item[$\Box$]  used appropriate sentence construction and word choice
	\item[$\Box$]  tested our manual to assure sixth to seventh grade reading level
	\item[$\Box$]  properly written and placed specific warnings and cautions
	\item[$\Box$]  has a durable distinctive cover
	\item[$\Box$]  will stand up to the conditions in which it will be used
	\item[$\Box$]  is constructed of non-shiny durable paper
	\item[$\Box$]  is laid out to make sections easy to find and update (tabs, binding, page numbering)
	\item[$\Box$]  uses white space and other highlighting techniques to focus user attention on important information
	\item[$\Box$]  is printed in at least 12 point type
	\item[$\Box$]  has clear, well-labeled graphics in key places to help users understand text
	\item[$\Box$]  is printed in proper contrast
\end{itemize}

\paragraph{We have:}

\begin{itemize} \itemsep -2pt
	\item[$\Box$]  tested the manual to make sure that our users can read, understand, and follow it
	\item[$\Box$]  have taken steps to make sure that our manual gets to our users 
\end{itemize}

\end{multicols}
\end{document}